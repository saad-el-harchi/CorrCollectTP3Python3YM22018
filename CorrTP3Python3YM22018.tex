\documentclass{article}

% packages pour le style que j'utilise pour écrire du python :
\usepackage{listings}
\usepackage{color}

% ici style pour code Python
\definecolor{codegreen}{rgb}{0,0.6,0}
\definecolor{codegray}{rgb}{0.5,0.5,0.5}
\definecolor{codepurple}{rgb}{0.58,0,0.82}
\definecolor{backcolour}{rgb}{0.95,0.95,0.92}
\lstset{
	language=python,
	backgroundcolor=\color{backcolour},   
    commentstyle=\color{codegreen},
    keywordstyle=\color{magenta},
    numberstyle=\tiny\color{codegray},
    stringstyle=\color{codepurple},
    basicstyle=\footnotesize,
    breakatwhitespace=false,         
    breaklines=true,                 
    captionpos=b,                    
    keepspaces=true,                 
    numbers=left,                    
    numbersep=5pt,                  
    showspaces=false,                
    showstringspaces=false,
    showtabs=false,                  
    tabsize=2
}
% ici termine style code Python


\title{Correction serie 3}
\author{Imane Sbai}
\date {\today}
\begin{document}
\begin{titlepage}
    \begin{center}
Correction s\'{e}rie 3 python
    \end{center}
\end{titlepage}

\section{exercice num1 :}
	\begin{center}
    		Abderrahmane :
	\end{center} 
	
\begin{lstlisting}
k=0
chaine=input("entrez une chaine de character : ")
for i in chaine:
	print(k, i)
	k=k+1
\end{lstlisting}
\section{exercice num2 :}

	\begin{center}
    		Ichrac :
	\end{center}
	
	
	\begin{center}
    		Loubna :
	\end{center} 
	

        \begin{center}
    		 Yahya Faouzi :
	\end{center}    	  
        	
    \begin{lstlisting}
    1->>>['a','b','b',2]
    2->>>[1,'b']
    3->>>['a','b',2]
    4->>>['a','b',2]
    \end{lstlisting}
         
	\begin{center}
    		Loic :
	\end{center} 
	
	\begin{lstlisting}
		1->>>8
		2->>>[1,2,3,4,2,3,4,5]
		3->>>1
		4->>>[[],1]
	\end{lstlisting}
	
	\begin{center}
    		Mohamed :
	\end{center}
	
	\begin{lstlisting}
    a=[ 3,  1,  4, 1, 5]
    b=[]
    for i in range (len(a)-1)
       b.append(a[i]+a[i+1])
    print(b)
    Résultat :
    B=[ 4,  5,  5, 6 ]
	\end{lstlisting}

	\begin{center}
    		Imane :
        \end{center} 
	
	\begin{lstlisting}
		Correction F:
	[!,1,2,!,4,5,!,7]

	\end{lstlisting}
	
    	\begin{center}
    		Moussa :
	\end{center} 
	
	\begin{lstlisting}
		Correction G:
		[1,4,1,4,1]
	\end{lstlisting}

    	\begin{center}
    		Hamza :
	\end{center} 
	\begin{lstlisting}
		correction H :
		liste=[1,3,4,5,8,2]
		pour i=1    : [1, 8, 4, 5, 8, 2]
		pour i=2    : [1, 8, 17, 5, 8, 2]
		pour i=3    : [1, 8, 17, 30, 8, 2]
		pour i=4    : [1, 8, 17, 30, 40, 2]
		
	\end{lstlisting}

	\begin{center}
    		Ayman :
	\end{center} 
	\begin{lstlisting}
	
liste=[[1]]
i=0
While i<4:
X=liste[i]
Y=[1]
J=1
While J<len(x):
Y.append(X[J-1] + X[J])
J=J+1
Y.append(1)
liste.append(Y)
i=i+1
Print(liste)
Resultat:
[[1],[1,1],[1,2,1],[1,3,3,1],[1,4,6,4,1]]
\end{lstlisting}
\section{exercice num3 :}
        \begin{center}
    		Saad El Harchi :
	\end{center}    	  
        	
    	\begin{lstlisting}
			entree=list()
			a=5
			while a <= 20  and a >= 0:
    			a=int(input("entrez une note comprise entre 0 et 20 : "))
    			if a <= 20 and a >= 0:
        			entree.append(a) 
			if len(entree) > 0 :
    			for i in range(len(entree)):
        			print("note {} : {}".format(i,entree[i]))
   			 a=0
   			 m=sum(entree)/len(entree)
    			for i in range(len(entree)):
        			a+=(entree[i]-m)**2
    
    			print("\n la note maximale : {}".format(max(entree)))
    			print("\n la note minimale : {}".format(min(entree)))
    			print("\n la moyenne : {}".format(m))
    			print("\n l'écart type : {}".format(((1/n)*a)**(1/2)))
			else:
    			print("acune note a calculer")
		\end{lstlisting}

\section{exercice num4 :}
	\begin{center}
    		Aymen, El Arbi, Mohamed :
	\end{center}
	\begin{lstlisting}
    def tri(a) :
        n=len(a)
        for i in range (n) :
            for j in range (n) :
                If a[i]>=a[j] and i>j
                    a[i]; a[i]=a[j];a[i]
	\end{lstlisting}
\end{document}
